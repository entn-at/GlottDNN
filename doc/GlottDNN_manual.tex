\documentclass[a4paper, 11pt]{article}
\usepackage[utf8]{inputenc}
\usepackage[toc,page]{appendix}
\usepackage{titlesec}

\usepackage[margin=1in,footskip=0.25in]{geometry}
\usepackage{color}
\usepackage{hyperref}
\hypersetup{
  colorlinks=true,
  linktoc=all,
  linkcolor=blue,
}

%\pagestyle{headings}
\usepackage{fancyhdr}
\pagestyle{fancy}
\fancyhf{}
\fancyhead[LE,RO]{\emph{\leftmark}}
\fancyhead[RE,LO]{\textsc{GlottDNN user manual}}
\fancyfoot[CE,CO]{\thepage}
\fancyfoot[L,R]{}

\titleformat{\section}
  {\Large\scshape}{\thesection}{1em}{}

  
\titleformat{\subsection}
  {\normalsize\bfseries}{\thesubsection}{2em}{}

\titleformat{\subsubsection}
  {\large\slshape}{\thesubsubsection}{2em}{}



\onecolumn

%\title{GlottDNN V.0.0.1 Manual}

%\author{Manu Airaksinen \& Lauri Juvela}


\begin{document}

\begin{titlepage}
\begin{center}
~\\[3cm]
\line(1,0){450} \\[1cm]
\Huge{\textsf{\textsc{GlottDNN}}}\\[1cm]
\huge{\textmd{A full-band glottal vocoder for\\
	 statistical parametric speech synthesis}}\\[0.5cm]
\line(1,0){450} \\[2cm] 
\Huge{\textsc{user manual}}\\
\huge{Version 0.1}\\[2cm] 

\Large{\today \\
      Manu Airaksinen \& Lauri Juvela\\}
	
\end{center}
\end{titlepage}
\stepcounter{page}

\tableofcontents

\newpage

\section{Introduction}

\subsection{Background}

\subsection{Licence and distribution}


\subsection{Authors \& contact}

\subsection{Acknowledgements}

\section{Quick-start guide}

\subsection{Installation}

This distribution intended for use in Unix-like systems that support the GNU autotools buildchain. Standard ./configure script and Makefile are provided for easy compilation.
%
The vocoder C++ code has the following library dependencies:

\begin{itemize}
	\item \texttt{GSL} (GNU scientific library), for basic linear algebra and FFT etc.
	\item \texttt{libsndfile}, for reading and writing audio files
	\item \texttt{libconfig}, for reading structured configuration files
\end{itemize}
%
Usually the best way to install the dependencies is with the system package manager. For example, in Ubuntu 14.04, install the packages \texttt{libgsl0-dev}, \texttt{libsndfile1-dev}, \texttt{libconfig++-dev}


Alternatively, you may download the source code for the libraries and compile them yourself. Make sure to add the built libraries into the \texttt{LD\_LIBRARY\_PATH},
 so that configure can find them. The libraries (and their documentation) are available at 
%
\begin{itemize}
	\item[] \href{https://www.gnu.org/software/gsl/}{https://www.gnu.org/software/gsl/}
	\item[] \href{http://www.mega-nerd.com/libsndfile/}{http://www.mega-nerd.com/libsndfile/}
	\item[] \href{http://www.hyperrealm.com/libconfig/ }{http://www.hyperrealm.com/libconfig/}
\end{itemize}
%
Additionally, this package uses the C++ wrappers for GSL provided at 
\href{http://gslwrap.sourceforge.net}{http://gslwrap.sourceforge.net}. The code is included in this distribution, but is set to build as a separate library.


We recommend to use the GitHub distributed version to get the latest updates, and for the ease of other people contributing to the development. First, clone the code to a suitable location with

\texttt{
git clone https://github.com/ljuvela/GlottDNN.git 
}

Then, if everything goes well, you should be able to compile with the basic 
\begin{itemize}
	 \item[]\texttt{./configure}
	 \item[]\texttt{make}
\end{itemize}
% 
If something went wrong, it might help to use autotools to regenerate the configure and make files.
\begin{itemize}
	 \item[]\texttt{autoreconf}
	 \item[]\texttt{automake --add-missing}
\end{itemize}


Python


\subsection{Configuration}

\subsection{Running Analysis}

\subsection{Running Synthesis}

\section{GlottDNN Analysis}

\subsection{Technical description}
TODO: Rundown of block diagram etc.

\subsection{Spectral (vocal tract) estimation}

\subsubsection{Quasi-closed phase analysis (QCP)}

\subsubsection{Iterative adaptive inverse filtering (IAIF)}

\subsubsection{Frequency-warped time-weighted linear prediction (WWLP)}

\subsubsection{Quadrature mirror filter (QMF) sub-band analysis}

\subsection{Pitch estimation}

\subsubsection{Fundamental frequency analysis}

\subsubsection{Glottal closure instant detection}


\subsection{Harmonic-to-noise ratio (HNR) estimation}

Harmonic-to-noise ratio describes the amount of aperiodicity in the glottal excitation signal

The excitation signal model is a simple additive one: excitation signal $g(t)$ consists of a deterministic harmonic component $h(t)$ and a stochastic noise component $n(t)$.

\begin{equation}
g(t) = h(t) + n(t)
\end{equation}

In practice, the HNR codes the ratio between upper and lower spectral envelopes, where the upper envelope is interpolated from the harmonic spectrum peaks, and the lower envelope is interpolated from spectral valleys between the peaks. 

Harmonic peak-picking is performed iteratively on the FFT magnitude at frame level. The first harmonic peak is taken as the highest FFT magnitude near $f_0$ 

For modeling, the HNR values are compressed in ERB bands

\begin{equation}
\mathrm{HNR}_i = \sum_{\omega \in \mathrm{ERB}_i}  20 \log \vert N(\omega) \vert - 20\log \vert H(\omega) \vert
\end{equation}

This is actually a noise-to-harmonic ratio

\subsection{Analysis features}

\section{GlottDNN Synthesis}

\subsection{Technical description}
TODO: Rundown of block diagram etc.

\subsection{Glottal excitation generation}

\subsubsection{DNN-based excitation generation}

\subsubsection{Pulses-as-features excitation}

\subsubsection{Library pulse excitation}

\subsection{Training of excitation DNN}

\section{Text-to-speech (TTS) pipeline integration}

\section{Configuration file explained}

\subsection{General shared parameters}

\begin{enumerate}
    \item[]  SAMPLING\_FREQUENCY : 
    Sampling frequency should match that of the wav file
    \item[]  FRAME\_LENGTH :
    Analysis frame length (in ms)
    \item[]   UNVOICED\_FRAME\_LENGTH :
    Analysis frame length in unvoiced frames. Shorter frames can better capture plosives and other impulse-like unvoiced events.
    \item[] F0\_FRAME\_LENGTH : Frame length used for fundamental frequency analysis. 
    \item[] FRAME\_SHIFT :
    Frame rate (in ms)
    \item[] LPC\_ORDER\_VT : 
    LPC order for the vocal tract filter
    \item[] LPC\_ORDER\_GLOT :
    LPC order for the glottal source
    \item[] HNR\_ORDER :
    Number of ERB bands for Harmonic-to-noise ratio
    \item[] DATA\_TYPE : Data type for saving and reading parameters. Valid types are "ASCII" / "DOUBLE" / "FLOAT"    
\end{enumerate}

\subsection{Pulse extraction related parameters}

\begin{enumerate}
    \item[]                                        
    MAX\_PULSE\_LEN\_DIFF :
    Percentage of how much pulse length can differ from F0. Pulses are searched iteratively until the nearest pulse fulfilling the length condition is found.     
    \item[]
    PAF\_PULSE\_LENGTH :
    Pulses-as-features length in samples. If interpolation is not used, this should be large enough to fit two pitch periods at the lowest F0.
    \item[] 
    USE\_PULSE\_INTERPOLATION :
    If true, two pitch-period pulses are interpolated to fill the feature vector. If false, the pulse is only centered at GCI.
    \item[] 
    USE\_WAVEFORMS\_DIRECTLY :
    If true, the speech waveform is used directly instead of the inverse filtered waveform.
    \item[]
    USE\_FOUR\_PERIOD\_PULSES :
    If true, Four pitch-periods are used instead of two.
    \item[] PAF\_WINDOW :
    Select the windowing function applied to the pulse at analysis. Valid options are 
    "NONE"/"HANN"/"COSINE"/"KBD"
    \item[]
    USE\_PAF\_ENERGY\_NORM :
    Normalize the pulse to unit energy.
\end{enumerate}


\newpage

\begin{appendices}
\section{Configuration file description}
\section{Excitation DNN file format}
\end{appendices}

\end{document}